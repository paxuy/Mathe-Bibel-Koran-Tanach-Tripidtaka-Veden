\documentclass[11pt,a4paper,oneside]{article}
\usepackage[utf8]{inputenc}
\usepackage[german]{babel}
\usepackage[T1]{fontenc}
\usepackage{lmodern}
\usepackage{amsmath}
\usepackage{amsfonts}
\usepackage{amssymb}

\usepackage{calc}
%\usepackage{showframe}

\usepackage{ulem}
\usepackage{cancel}
\usepackage{float}
\usepackage{subcaption}
\usepackage{hyperref}
\hypersetup{
    colorlinks,
    citecolor=black,
    filecolor=black,
    linkcolor=black,
    urlcolor=black
}
\usepackage[a4paper, left=2cm,right=2cm,top=2cm,bottom=2cm]{geometry}

\def\AUTHOR{Paul Schillinger}
\author{\AUTHOR}
\def\MYTITLE{Mathe-Bibel-Koran-Tanach-Tripi\d{t}aka-Veden v1.0}
\title{\MYTITLE}

\usepackage{fancyhdr}
\pagestyle{fancy}
\setlength{\headheight}{14pt}
\rhead[]{\AUTHOR}
\lhead[]{\MYTITLE}

\newcommand{\sidebyside}[3]{
\newlength{\lenA}
\setlength{\lenA}{#1\linewidth}
\begin{figure}[h]
\begin{subfigure}[t]{0.5\linewidth minus 1\lenA}
#2
\end{subfigure}%
\hspace{2\lenA}
\begin{subfigure}[t]{0.5\linewidth minus 1\lenA}
#3
\end{subfigure}
\end{figure}
}


\newcommand{\sidebysideB}[2]{
\begin{figure}[H]
\begin{subfigure}[t]{0.5\linewidth}
#1
\end{subfigure}%
\begin{subfigure}[t]{0.5\linewidth}
#2
\end{subfigure}
\end{figure}
}


\begin{document}
\maketitle
\tableofcontents
\newpage
\section*{Grundlagen der Mathematik}
\addcontentsline{toc}{section}{Grundlagen der Mathematik}
\section{Grundrechenarten}
\begin{figure}[h]
	\begin{subfigure}[t]{0.5\linewidth}
\subsection{Addition}
Summand $+$ Summand $=$ Summe
\subsection{Subtraktion}
Minuend $-$ Subtrahend $=$ Differenz
	\end{subfigure}%
	\begin{subfigure}[t]{0.5\linewidth}
\subsection{Multiplikation}
Faktor $\cdot\,\mid\,\times$ Faktor $=$ Produkt
\subsection{Division}
Dividend\: $:\:\mid\,\div\,\mid\,/$ Divisor $=$ Quotient
	\end{subfigure}
\end{figure}
\vspace{-1em}
\section{Zahlenbereiche}
\begin{figure}[h]
	\begin{subfigure}[t]{0.5\linewidth}
\subsection{Natürliche Zahlen}
\subsubsection*{ohne Null}
Mathematisches Symbol: $\mathbb{N}$
\\
Beispiele: $1;\,2;\,3;\,\ldots$
\subsubsection*{mit Null}
Mathematisches Symbol: $\mathbb{N}_{0}$
\\
Beispiele: $0;\,1;\,2;\,3;\,\ldots$
\subsection{Ganze Zahlen}
Mathematisches Symbol: $\mathbb{Z}$
\\
Beispiele: $\ldots;\,-2;\,-1;\,0;\,1;\,2;\,\ldots$
\subsection{Gebrochene Zahlen}
Mathematisches Symbol: $\mathbb{Q}^{+}$ oder $\mathbb{Q}^{*}$
\\
Beispiele: $\frac{1}{2};\,1;\,\frac{2}{7};\,\ldots$

	\end{subfigure}%
		%\hspace{0.04\linewidth}
	\begin{subfigure}[t]{0.5\linewidth}
\subsection{Rationale Zahlen}
Mathematisches Symbol: $\mathbb{Q}$
\\
Beispiele: $\ldots;\,\frac{1}{2};\,-\frac{1}{2};\,1;\,-1;\,\frac{2}{7};\,-\frac{2}{7};\,\ldots$
\subsection{Reele Zahlen}
Mathematisches Symbol: $\mathbb{R}$
\\
Beispiele: $\sqrt{2};\,\pi;\,e;\,\ldots$
\subsection{Komplexe Zahlen}
Mathematisches Symbol: $\mathbb{C}$
\\
Beispiele: $i;\,7+3i;\,3-4i;\,\ldots$
	\end{subfigure}
\end{figure}
\vspace{-1em}

\section{Rechengesetze}
\subsection{Rechenreihenfolge}
\begin{enumerate}
\item Klammer
\item Potenz
\item Punkt (Multiplikation und Division)
\item Strich (Addition und Subtraktion)
\item Links nach Rechts
\end{enumerate}

\newpage
\subsection{Assoziativgesetz (Verknüpfungsgesetz)}
Gilt nur bei Addition und Multiplikation, nicht bei Subtraktion und Division.\\
\vspace{-2em}
\begin{table}[h]
\begin{tabular}{@{}lc}
\rule{0pt}{1em}Addition: & $\left(a+b\right)+c=a+\left(b+c\right)=a+b+c$\\
\rule{0pt}{1em}Multiplikation: & $\left(a\cdot b\right)\cdot c=a\cdot \left(b\cdot c\right)=a\cdot b\cdot c$
\end{tabular}
\end{table}
\vspace{-1.5em}

\subsection{Distributivgesetz (Verteilungsgesetz)}
\vspace{-1em}
\sidebysideB{
Multiplikation:\\
$a\cdot\left(b\pm c\right)=a\cdot b\pm a\cdot c$\\
$\left(a\pm b\right)\cdot c=a\cdot c\pm b\cdot c$
}{
Division:\\
\sout{$a\div\left(b\pm c\right)=a\div b\pm a\div c$}\\
$\left(a\pm b\right)\div c=a\div c\pm b\div c$\quad{\footnotesize (s. \ref{Bruchgesetze}: Add/Sub)}
}
\subsection{Kommutativgesetz (Vertauschungsgesetz)}
Gilt nur bei Addition und Multiplikation, nicht bei Subtraktion und Division.
\sidebysideB{Addition:\qquad$a+b=b+a$}{Multiplikation:\qquad$a\cdot b=b\cdot a$}

\section{Brüche}
$\dfrac{\text{Dividend}}{\text{Divisor}} = \dfrac{\text{Zähler}}{\text{Nenner}} = \dfrac{\text{Z}}{\text{N}} = (\text{Z})\div(\text{N})\qquad \text{N}\neq0$, da nicht definiert

\subsection{Kürzen}
\vspace{-1ex}
\begin{table}[H]
\begin{tabular}{ll}
$\dfrac{a\cdot c}{b\cdot c} = \dfrac{a}{b}$\ \qquad  &
\begin{tabular}{rl}
\textit{Merkspruch:} & \begin{tabular}[c]{@{}l@{}}Differenzen und Summen\:\\ \:\,kürzen nur die Dummen.\end{tabular}
\end{tabular}
\end{tabular}
\end{table}

\subsection{Bruchgesetze}
\label{Bruchgesetze}
\vspace{-1em}
\begin{table}[H]
\begin{tabular}{@{}lc@{\hskip 2em}c}
\begin{tabular}[c]{@{}l@{}}Addition/Subtraktion\\ (von gleichnamigen Brüchen)\end{tabular} & $\dfrac{a}{n}\pm\dfrac{b}{n}=\dfrac{a\pm b}{n}$ & $\dfrac{1}{3}\pm\dfrac{2}{3}=\dfrac{1\pm 2}{3}$\\
\rule{0pt}{2em}Multiplikation mit natürlichen Zahlen & $\dfrac{a}{b}\cdot n=\dfrac{a\cdot n}{b}$ & $\dfrac{1}{2}\cdot 3=\dfrac{1\cdot 3}{2}=\dfrac{3}{2}$\\
\rule{0pt}{2em}Division durch natürliche Zahlen & $\dfrac{a}{b} \div n=\dfrac{a}{b\cdot n}$ & $\dfrac{1}{2} \div 3=\dfrac{1}{2\cdot 3}=\dfrac{1}{6}$\rule[-1.5em]{0pt}{0pt}\\
\end{tabular}
\newline
\begin{tabular}{@{}lc@{\hskip 2em}c}\hline
\rule{0pt}{2em}Addition/Subtraktion & $\dfrac{a}{b}\pm\dfrac{c}{d}=\dfrac{a\cdot d \pm b\cdot c}{b\cdot d}$ & $\dfrac{1}{2}\pm\dfrac{3}{4}=\dfrac{1\cdot 4 \pm 2\cdot 3}{2\cdot 4}=\dfrac{4\pm6}{8}$\\
\rule{0pt}{2em}Multiplikation & $\dfrac{a}{b}\cdot\dfrac{c}{d}=\dfrac{a\cdot c}{b\cdot d}$ & $\dfrac{1}{2}\cdot\dfrac{3}{4}=\dfrac{1\cdot 3}{1\cdot 4}=\dfrac{3}{4}$\\
\rule{0pt}{2em}Division & $\dfrac{a}{b}\div\dfrac{c}{d}=\dfrac{\frac{a}{b}}{\frac{c}{d}}=\dfrac{a}{b}\cdot\dfrac{d}{c}=\dfrac{a\cdot d}{b\cdot c}$ & $\dfrac{1}{2}\div\dfrac{3}{4}=\dfrac{1}{2}\cdot\dfrac{4}{3}=\dfrac{1\cdot 4}{2\cdot 3}=\dfrac{4}{6}$\\
\end{tabular}
\end{table}

\subsection{Potenzgesetze}
\vspace{-2em}
\begin{figure}[H]
\begin{subfigure}[t]{0.5\linewidth}
\begin{table}[H]
\begin{tabular}{@{}lc}
\multicolumn{2}{@{}c}{\textbf{Gleiche Basis}}\\
\rule{0pt}{1.5em}Multiplikation & $x^{a}\cdot x^{b}=x^{a+b}$\\
\rule{0pt}{1.5em}Division & $x^{a}\div x^{b}=\dfrac{x^{a}}{x^{b}}=x^{a-b}$\\
\rule{0pt}{1.5em}Potenzen potenzieren & $\left(x^{a}\right)^{b}=x^{a\cdot b}$\\
\multicolumn{2}{@{}c}{\rule{0pt}{2em}\textbf{Unterschiedliche Basis}}\\
\rule{0pt}{1.5em}Multiplikation & $a^{n}\cdot b^{n}=\left(a\cdot b\right)^{n}$\\
\rule{0pt}{1.5em}Division & $a^{n}\div b^{n}=\dfrac{a^{n}}{b^{n}}=\left(\dfrac{a}{b}\right)^{n}$\\
\end{tabular}
\end{table}
\end{subfigure}%
%\hspace{0.04\linewidth}
\begin{subfigure}[t]{0.5\linewidth}
\begin{table}[H]
%\flushright
\begin{tabular}{@{}lc}
\multicolumn{2}{@{}c}{\textbf{Negative Basis}}\\
\rule{0pt}{2em}\begin{tabular}[c]{@{}l@{}}Exponent gerade\\ Vorzeichen verschwindet\end{tabular} & $ \left(-2\right)^{2}=4 $\\
\rule{0pt}{2em}\begin{tabular}[c]{@{}l@{}}Exponent ungerade\\ Vorzeichen bleibt\end{tabular}  & $ \left(-2\right)^{3}=-8 $\\
\multicolumn{2}{@{}c}{\rule{0pt}{2em}\textbf{Besondere Exponenten}}\\
\rule{0pt}{1.5em}Exponent = 0 & $x^{0}=1$\\
\rule{0pt}{1.5em}Negative Exponenten & $x^{-n}=\dfrac{1}{x^{n}}$\\
\rule{0pt}{1.5em}Brüche als Exponent & $x^{\frac{m}{n}}=\sqrt[n]{x^{m}}$\\
\rule{0pt}{1.5em}Negative Brüche als Exponent & $x^{-\frac{m}{n}}=\dfrac{1}{\sqrt[n]{x^{m}}}$\\
\end{tabular}
\end{table}
\end{subfigure}
\end{figure}
\vspace{-2em}
sdfsf
\newpage
\section{Potenzen}
\sidebysideB{
$\text{Basis}^{\text{Exponent}}=a^{b}=\underbrace{a\cdot a\cdot a\cdot\ldots\cdot a}_{b\text{ Faktoren}}$}{$a^{b}=x\quad\Leftrightarrow\quad a=\sqrt[b]{x}$}

\subsection{Binomische Formeln}
\vspace{-1em}
\begin{table}[H]
\begin{tabular}{@{}lc}
\rule{0pt}{1.5em}1. Binomische Formel (Plus-Formel) & $\left(a+b\right)^{2}=a^{2}+2\cdot a\cdot b+b^{2}$\\
\rule{0pt}{1.5em}2. Binomische Formel (Minus-Formel) & $\left(a-b\right)^{2}=a^{2}-2\cdot a\cdot b+b^{2}$\\
\rule{0pt}{1.5em}3. Binomische Formel (Plus-Minus-Formel) & $\left(a+b\right)\cdot\left(a-b\right)=a^{2}-b^{2}$\\
\end{tabular}
\end{table}


%\newpage







\vspace{-1em}
\end{document}





\begin{comment}
\begin{figure}
\begin{subfigure}[h]{0.5\linewidth}

\end{subfigure}%
%\hspace{0.04\linewidth}
\begin{subfigure}[h]{0.5\linewidth}

\end{subfigure}
\end{figure}
\vspace{-1em}
\end{comment}
